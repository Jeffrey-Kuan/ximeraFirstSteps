\documentclass{ximera}

\title{Workflow}
\author{Bart Snapp}

\begin{document}
\begin{abstract}
    Suggestions for working on Ximera documents.
\end{abstract}
\maketitle

Authors own their content in Ximera and work on their own machine.
Nevertheless, with the assistance of Visual Studio Code and Docker, we are able
to provide a common deployment environment for MacOS, Windows, and Linux.

\paragraph{Docker}

Docker is a development utility that allows us to set up a computer within your
computer. You must remember to start Docker before you deploy. You can just let
it run in the background.

\paragraph{Visual Studio Code}
is a popular text-editor. It is has strong \LaTeX, GitHub,
and Docker integration. It provides a UNIX-like command line interface for
Windows machines via WSL. A typical workflow would be to:
\begin{enumerate}
    \item Open Docker and minimize the window.
    \item Open Visual Studio Code, do File $\to$ Open Folder, and select the
          folder of your git repository.
    \item To open files, do \verb!Ctrl-p! and start typing file names. Any file
          committed to your git repository will be found, and files in
          \verb!.gitignore!, will not be shown.
\end{enumerate}

Commands like \verb!Ctrl-~! and \verb!Ctrl-p! \verb!Ctrl-Shift-p!

Docker commands like
\verb!docker ps!

\subsection{Extensions for Visual Studio Code}

\LaTeX

git

docker

markdown

GitHub and ssh

Convert https repo to ssh repo

\section{Using \texttt{git}}

ADDING GRAPHIC VIA GIT

All files need to be commited to the GitHub repository.
Now return to the terminal window, which should still be in the
directory
\verb!theExampleCourse!. You need to add the file (and the directory
containing
it) to git and commit your changes, and also tell it who you are.
\begin{verbatim}
git add anExampleCourse.tex
git commit -m "this is my first course"
\end{verbatim}

Every time you add or change files, you will need to run \verb!git add!
and
\verb!git commit -m "short description of changes"! to commit the
changes to
the server. The description of changes is necessary to commit, as is
the
\verb! -m!. If you are changing multiple files, you can use
\verb!git add -u!.

Now add this file to git and commit your changes. Back in the terminal,
change to the directory \verb!anExampleCourse!
and execute the following commands
\begin{verbatim}
git add theFirstActivity.tex
git commit -m "Added first activity file"
\end{verbatim}

\section{Deploying and debugging}

\begin{verbatim}
./scripts/xmlatex -i bash
xake -v compile FILE-NAME.tex
more FILE-NAME.tex.log
pdflatex FILE-NAME.tex
\end{verbatim}

Slowly move \verb!\end{document}! down a document.

\end{document}