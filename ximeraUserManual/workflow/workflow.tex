\documentclass{ximera}

\title{Workflow}
\author{Bart Snapp}

\begin{document}
\begin{abstract}
    Suggestions for working on Ximera documents.
\end{abstract}
\maketitle

Authors own their content in Ximera and work on their own machine.
Nevertheless, with the assistance of Visual Studio Code and Docker, we are able
to provide a common deployment environment for MacOS, Windows, and Linux.


\section{Docker and Visual Studio Code}

\paragraph{Docker}

Docker is a development utility that allows us to set up a computer within your
computer. You must remember to start Docker before you deploy. You can just let
it run in the background. You can check the status of Docker by running
\verb!docker ps! in a terminal session.

\paragraph{Visual Studio Code}
is a popular text-editor. It is has strong \LaTeX, GitHub,
and Docker integration. It provides a UNIX-like command line interface for
Windows machines via WSL. A typical workflow would be to:
\begin{enumerate}
    \item Open Docker and minimize the window.
    \item Open Visual Studio Code, do File $\to$ Open Folder, and select the
          folder of your git repository.
    \item To open files, do \verb!Ctrl-p! and start typing file names. Any file
          committed to your git repository will be found, and files in
          \verb!.gitignore!, will not be shown.
    \item To run a special command (like search and replace) do
          \verb!Ctrl-Shift-p!, and search for the command.
    \item To open a terminal, use \verb!Ctrl-~!.
\end{enumerate}

\section{Using \texttt{git}}

When you use GitHub, you are writing code in public. You don't want randos to
be able to change your code, and they cannot.  But that same protection makes
it difficult for you to develop. So we need to set up a bit of security, called
SSH, so that you can authenticate but others cannot.
Once we are finished with that, we will give you some basic \texttt{git} commands. 


\paragraph{Setting up SSH} GitHub requires some security. One way to do this is
to add an SSH key. To check to see if this is already done, open Visual Studio
Code, open a (WSL) terminal with \verb!Ctrl-~! and enter
\begin{verbatim}
git remote -v
\end{verbatim}
if it responds with something like: If it responds with \verb!git@github.com:STUFF! then you
are good and can skip this. If not, and it responds with  \verb!https://github.com/STUFF! the you
need to change to SSH, read on!

First we must create the key. There are instructions on \link[GitHub's pages]{https://docs.github.com/en/authentication/connecting-to-github-with-ssh/generating-a-new-ssh-key-and-adding-it-to-the-ssh-agent}
Once the key is created, go to your personal GitHub page, like
\verb!https://github.com/bartsnapp! and:
\begin{enumerate}
    \item  Click on \textit{Settings}, it lives in the circular picture in
          upper right hand corner.
    \item Click on \textit{SSH and GPG keys}, it's on the left.
    \item Click the green button \textit{New SSH key}
    \item Name it \textit{My Ximera Key}.
    \item Paste in your Key.
\end{enumerate}
Now when you clone a repository, you will use the SSH option.  To convert a
repository from the HTTPS option to SSH,
\begin{verbatim}
git remote set-url origin git@github.com:STUFF
\end{verbatim}




\paragraph{Basic commands}

\begin{verbatim}
git add -u
git commit -m "this is my change"
git push
\end{verbatim}


\begin{verbatim}
git add -u && git commit -m "this is my change" && git push
\end{verbatim}

\begin{verbatim}
git add anExampleCourse.tex
git commit -m "this is my first course"
git push
\end{verbatim}

\paragraph{Adding images and PDFs with \texttt{git}}

All files need to be commited to the GitHub repository.
Now return to the terminal window, which should still be in the
directory
\verb!theExampleCourse!. You need to add the file (and the directory
containing
it) to git and commit your changes, and also tell it who you are.
\begin{verbatim}
git add anExampleCourse.tex
git commit -m "this is my first course"
\end{verbatim}

Every time you add or change files, you will need to run \verb!git add!
and
\verb!git commit -m "short description of changes"! to commit the
changes to
the server. The description of changes is necessary to commit, as is
the
\verb! -m!. If you are changing multiple files, you can use
\verb!git add -u!.

Now add this file to git and commit your changes. Back in the terminal,
change to the directory \verb!anExampleCourse!
and execute the following commands
\begin{verbatim}
git add theFirstActivity.tex
git commit -m "Added first activity file"
\end{verbatim}

\section{Deploying and debugging}

\begin{verbatim}
./scripts/xmlatex -i bash
xake -v compile FILE-NAME.tex
more FILE-NAME.tex.log
pdflatex FILE-NAME.tex
\end{verbatim}

Slowly move \verb!\end{document}! down a document.

\end{document}