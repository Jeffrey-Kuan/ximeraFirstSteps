\documentclass{ximera}

\title{Workflow}
\author{Bart Snapp}

\begin{document}
\begin{abstract}
    Suggestions for working on Ximera documents.
\end{abstract}
\maketitle

\section{Work on your machine}



\section{Docker}



\section{Visual Studio Code}
Commands like \verb!Ctrl-~! and \verb!Ctrl-p! \verb!Ctrl-Shift-p!

Docker commands like
\verb!docker ps!

\subsection{Extensions for Visual Studio Code}

\LaTeX

git

docker

markdown



GitHub and ssh

Convert https repo to ssh repo




\section{Using \texttt{git}}

All files need to be commited to the GitHub repository.
Now return to the terminal window, which should still be in the
directory
\verb!theExampleCourse!. You need to add the file (and the directory
containing
it) to git and commit your changes, and also tell it who you are.
\begin{verbatim}
git add anExampleCourse.tex
git commit -m "this is my first course"
\end{verbatim}

Every time you add or change files, you will need to run \verb!git add!
and
\verb!git commit -m "short description of changes"! to commit the
changes to
the server. The description of changes is necessary to commit, as is
the
\verb! -m!. If you are changing multiple files, you can use
\verb!git add -u!.



Now add this file to git and commit your changes. Back in the terminal,
change to the directory \verb!anExampleCourse!
and execute the following commands
\begin{verbatim}
git add theFirstActivity.tex
git commit -m "Added first activity file"
\end{verbatim}

\section{Deploying and debugging}


\begin{verbatim}
./scripts/xmlatex -i bash
xake -v compile FILE-NAME.tex
pdflatex FILE-NAME.tex
\end{verbatim}

\end{document}