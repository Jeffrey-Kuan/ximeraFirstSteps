\documentclass{ximera}
% \handouttrue
%%% Where to find images
\graphicspath{  %% When looking for images,
{./}            %% look first at your level,
{./basics/}     %% then in this folder,
}    
%\addPrintStyle{..}

\author{Wim Obbels \and Bart Snapp}
\title{Second  Activity}

\begin{document}
\begin{abstract}
    Another simple Ximera activity.
\end{abstract}
\maketitle
\label{xim:secondActivity}

A number of environments are provided by the Ximera document class. You can find a list of them HERE GIVE A LINK TO SOMETHING

When writing sections of a textbook, another way to include questions is in the solution/explanation.

\begin{example}[Population Counts] %https://worldpopulationreview.com/states/states-by-race
    The Midwest of the United States consists of $12$ states. We can
    express the $2023$
    \link[demographics]{https://worldpopulationreview.com/states/states-by-race}
    of each state as a vector represented by an ordered tuple. The
    ordered tuple for Ohio looks like:
    \[
    \vec{p}_{\texttt{OH}} = (\underset{\text{White}}{9394878},\underset{\begin{smallmatrix}\text{African}\\ \text{American}\end{smallmatrix}}{1442655},\underset{\begin{smallmatrix}\text{Native}\\\text{American}\end{smallmatrix}}{20442},\underset{\text{Asian}}{268527},\underset{\text{Hawaiian}}{3907},\underset{\text{Other}}{544866}).
    \]
    The ordered tuples for each state in the Midwest looks like:
  \[
  \begin{aligned}
    \vec{p}_{\texttt{IA}} &= (2806418,117035,10538,79296,3941,132783)\\
    \vec{p}_{\texttt{IL}} &= (8874067,1796660,33972,709567,5196,1296702)\\
    \vec{p}_{\texttt{IN}} &= (5510354,631923,14030,158705,2205,379676)\\
    \vec{p}_{\texttt{KA}} &= (2416165,165837,22278,87093,2344,218902)\\
    \vec{p}_{\texttt{MI}} &= (7735902,1360149,50035,316844,3117,507860)\\
    \vec{p}_{\texttt{MN}} &= (4572149,359817,54558,275242,2201,336199)\\
    \vec{p}_{\texttt{MO}} &= (4978046,698043,24274,123810,8887,291100)\\
    \vec{p}_{\texttt{ND}} &= (651470,23959,39165,11979,1004,32817)\\
    \vec{p}_{\texttt{NE}} &= (1641256,91896,16875,47944,1235,124620)\\
    \vec{p}_{\texttt{OH}} &= (9394878,1442655,20442,268527,3907,544866)\\
    \vec{p}_{\texttt{SD}} &= (735228,18836,74975,12413,544,37340)\\
    \vec{p}_{\texttt{WI}} &= (4895065,367889,48674,163396,2672,329279)
  \end{aligned}
  \]
  \begin{enumerate}
  \item What are the combined demographics of the states Michigan, Ohio,
    and Indiana?
  \item Suppose that the annual percentage growth rate of Ohio is
    currently $0.1\%$. Assuming this is even across all demographics,
    what might the population data look like for Ohio in $2025$?
  \end{enumerate}
  \begin{explanation}
    We'll use the properties of vectors to solve this problem.
    \begin{enumerate}
    \item To find the combined demographics of Michigan, Ohio, and
      Indiana we compute
      \[
      \vec{p}_{\texttt{MI}} + \vec{p}_{\texttt{OH}} + \vec{p}_{\texttt{IN}} = \left(\answer[given]{22641134},\answer[given]{3434727},\answer[given]{84507},744076,9229,1432402\right).
      \]
    \item To find the population demographics of Ohio in \textit{two}
      years, we use the scalar, $1.001^2$, to find:
      \[
        1.001^2 \vec{p}_{\texttt{OH}}
        =\left(\answer[given]{9413677},\answer[given]{1445542},\answer[given]{20483},269064,3915,545956
        \right).
      \]
      % \[
      % 1.001 \vec{p}_{\texttt{OH}} =\left(\answer[given]{9404273},\answer[given]{1444098},\answer[given]{20462},268796,3911,545411 \right)
      % \]
    \end{enumerate}
  \end{explanation}
  \end{example}

Examples use \verb|\begin{example}|.
By default, examples also provide the correct answer in the handout version,
while that is not the case for the exercises.

% \renewcommand{\choiceminimumverticalsize}{\vphantom{$\sqrt{2}$}} 

\begin{example}[Simple examples of absolute values]

    %\begin{xmmulticols}
    \begin{enumerate}
        \item $|5|=5$ and $|-5|=5$
        \item $|\sqrt{2}-1| =
              $\wordChoice{\choice[correct]{$\sqrt{2} -
                          1$}\choice{$1-\sqrt{2}$}}
        \item $|1-\sqrt{2}| =
              $\wordChoice{\choice[correct]{$\sqrt{2} -
                          1$}\choice{$1-\sqrt{2}$}}
        \item $|2-\sqrt{2}| = $\wordChoice{\choice{$\sqrt{2} -
                          2$}\choice[correct]{$2-\sqrt{2}$}}
    \end{enumerate}
    %\end{xmmulticols}
\end{example}

\end{document}