\documentclass{ximera}
% \handouttrue
%%% Where to find images
\graphicspath{  %% When looking for images,
{./}            %% look first at your level,
{./basics/}     %% then in this folder,
}    
%\addPrintStyle{..}

\author{Wim Obbels \and Bart Snapp}
\title{First Activity}

\begin{document}
\begin{abstract}
    A simple Ximera activity.
\end{abstract}
\maketitle
\label{xim:firstActivity}

This is an example of a Ximera activity that may contain discussion as well as
questions. We give examples
with some useful \hyperref[xim:ximeraEnvironments]{environments} and
\hyperref[xim:ximeraCommands]{commands}.

The first thing to note is that the title of the document is the same as the
title of the source file, and this is the same as the title of the folder
containing this files.
This is \textbf{not necessary}; however adhearing to a standard like this will
make your work easier to debug and share with others for years to come.

% Demo: small adhoc differences between PDF and HTML version
\pdfOnly{
    \begin{remark}
        It is advisable to also view the Online version, where this remark
        refers to the PDF.
        \ifhandout
            By the way, you are using the \textit{handout} PDF, which does
            \textbf{not} contain answers. \\
            There is also a so-called \textit{standard} PDF \textit{which does
                contain answers and hints}.
        \else
            You are, by the way, using the so-called \textit{standard} PDF,
            which \textbf{contains the answers} to the exercises. \\
            There is also a \textit{handout} PDF \textit{without the answers}.
        \fi
    \end{remark}
}
\begin{onlineOnly}
    \begin{remark}
        It is advisable to also view the PDF (GIVE LINK TO A DEPLOYED VERSION)
        version, where this remark
        suggests to
        consult the Online version.
    \end{remark}
\end{onlineOnly}

Use \verb|\begin{definition}| for definitions, and \verb|\begin{exercise}| for
exercises. Since Ximera allows for immediate feedback, we suggest following
definitions like this one by a quick exercise, or question.

\begin{definition}\label{showcase:absolutevalue}
    The \textbf{absolute value} of a real number $a$, denoted by $|a|$, is
    \[
        |a| = \begin{cases}
            a  & \text{if  $a \geq 0$} \\
            -a & \text{if  $a<0$.}
        \end{cases}
    \]
\end{definition}
Now students can check their understanding:
\begin{exercise}
    \begin{enumerate}
        \item   $|2-5|	   = \answer{3}$
        \item $|5-2|= \answer[onlineshowanswerbutton]{3}$
        \item	$|5-\sqrt{2}| = \answer[onlinenoinput]{3.58578643763}$
        \item  $|1-\sqrt{2}| = $\wordChoice{\choice[correct]{$\sqrt{2} -
                          1$}\choice{$1-\sqrt{2}$}}
    \end{enumerate}
    %\end{multicols}
\end{exercise}

A number of environments are provided by the Ximera document class. You can find a list of them on \href{https://ximera.osu.edu/testing/examples/problems/problem}{https://ximera.osu.edu/testing/examples}


\end{document}