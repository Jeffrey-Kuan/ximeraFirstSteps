\documentclass{ximera}
\begin{document}
\title{Ximera simple exercises}
\begin{abstract}
\end{abstract}
\maketitle

\begin{exercise}
    Let $x$ be the number of people
    out of $100$ that LOVE Ximera.\\
    
    Find the value of $x$.
    \[
        x = \answer{100}
    \]
\end{exercise}

\begin{exercise}
    Ximera is so awesome because it:
    \begin{multipleChoice}
        \choice{Feels like doing taxes}
        \choice{Writing a book by hand}
        \choice[correct]{A walk with free ice cream} 
        \choice{Solving a puzzle blindfolded}
    \end{multipleChoice}
\end{exercise}

\begin{exercise}
    Why is Ximera the best thing since the chalkboard?
    \begin{selectAll}
        \choice[correct]{Turns \LaTeX\ into online materials easily}
        \choice[correct]{Boosts student engagement}
        \choice{Makes coffee and hugs you}
        \choice[correct]{Open-source and free}
    \end{selectAll}
\end{exercise}

\begin{exercise}
Ximera uses \wordChoice{
\choice{Haskell}
\choice[correct]{\LaTeX}
\choice{XML}
} as an authoring language.
\end{exercise}


\begin{exercise}
    We can have follow up questions. First compute
    \[
    1+2 = \answer{3}
    \]
    \begin{exercise}
        Knowing that the answer to the previous question is $3$, what is the previous answer plus one?
        \[
        \answer{4}
        \]
    \end{exercise}
\end{exercise}


\begin{exercise}
    We can have follow up questions. First compute
    \[
    1+2 = \answer{3}
    \]
    \begin{problem}
        Knowing that the answer to the previous question is $3$, what is the previous answer plus one?
        \[
        \answer{4}
        \]
    \end{problem}
\end{exercise}

\begin{exercise}
    We can have hints.
    \[
    1+2 = \answer{3}
    \]
    \begin{hint}
What is $1+ 1+1$? 
    \end{hint}
\end{exercise}



\end{document}
